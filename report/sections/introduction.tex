\documentclass[main.tex]{subfiles}


\begin{document}
\chapter{Introduction}

Since the early establishment of the cell theory in the 19th century, which acknowledged the cell as the building block of life ~\cite{Muller-Wille2010}, biologist have sought to explain the underlying principles, and their application to improve health. In the course of many decades of research, they achieved great progress, thanks to the advances made in Light Microscopy, which revolutionized the field. This progress continues till this day with higher budgets, more minds and sophisticated tools. The complexity and heterogeneity of the modern imaging makes the processing and analyzing of such data very difficult. Computer methods became key for keeping the pace of progress, and make use of the available data.\\

One major problem in biomedical image analysis is Image segmentation. Recent progress in Deep learning attain performance unprecedented, notably with architectures like UNet~\cite{Ronneberger2015} which allowed to obtain consistently good segmentations comparable to expert. Particularly, the problem of cell segmentation, which is the task of assigning a cell instance identity to each pixel of the image, remain a major challenge. The analysis of cell cultures and tissue at single cell level is a central in several pathology pipelines~\cite{Boquet-Pujadas2021}. Features like Morphology, Density, nucleus to cytoplasm ratio, average size, shapes, can be useful for cancer grades assessment, predicting treatment effectiveness, identification of development lineage of dividing cells, cell tracking~\cite{Boquet-Pujadas2021}, classifying phenotypes, profiling treatments, counting cells, localizing proteins \dots. Therefore, accurately segment cells can significantly contribute to the development of clinical and medical research, and free up the biologist to work on more important things.\\


Cell segmentation received increasing attention in the past years. Many biomedical image analysis tools exist in the literature that can segment cells in images~\cite{Boquet-Pujadas2021}. Cells used in experimentation lost their contact inhibition and can pile up to each other, making the detection of single cells extremely challenging. Machine Learning methods are more and more used to provide automated tools, but they often require a large number of fully annotated images which is very tedious and time-consuming, recent works look to minimize the training data used. Even with these efforts, Most of the currently applied methods require expertise to select the algorithm that suits the problem and fine tune the parameters for every experiment.\\

A common yet simple method is to apply an intensity threshold (local or global) for separating the foreground from the background~\cite{Meijering2012}, but in practice it produces poor segmentations and applied only as a first step in the pipeline.\par
Feature based approaches  such as blob detector (Laplacian of Gaussian)~\cite{Hafiz2020} or Wavelet transform  followed by a threshold can detect regions corresponding to cells~\cite{Boquet-Pujadas2021}, but in practice are used as a seed initialization.Another common tools are morphological filters such as erosion, dilation, opening, closing, which can be used in preprocessing as well as post-processing~\cite{Meijering2012}.\par
Active contours (snakes) have proven their usefulness in medical image analysis\cite{Boquet-Pujadas2021}, but the simplest models often don't work well on our problem, different variations exist such as B-spline Snakes \cite{Brigger2000}, Hermite snakes~\cite{Uhlmann2016}, Gas of circle model~\cite{Molnar2016}, but are difficult to tune and require expert knowledge for designing a good energy function adapted to the type of problem encountered, are computationally expensive, and require good initialization.\par
Recent works in Deep learning use Bottom-up~\cite{Hafiz2020} approach which try to classify each pixel then group pixels to form instances (using watershed, by finding connected components, or by using CRF (Conditional Random fields)\cite{Arnab2018}).\par
Alternatively top-down~\cite{Hafiz2020} approach try to localize first the objects using a rough shape then refines it, typically use Mask-RCNN~\cite{Hafiz2020, he2017mask, johnson2018adapting}, or more recently SoloV2 \cite{Wang2020}, achieved state of the art results, and outperform their counterpart bottom up approaches in most of the cases which tend to have less segmentation accuracy. These top-down methods have in common that they use a Non Maximum Suppression step to avoid detecting the same object multiple times, but given the nature of cell segmentation where multiple cells overlap and interact with each other, bonding box predictions doesn't work as well as expected.\par
One approach in particular that became popular is Stardist~\cite{Schmidt2018} due to its simplicity and efficiency. StarDist directly predicts star convex polygons (without refinement step) using a UNet~\cite{Ronneberger2015} architecture to predict a fixed number of points on an object's contour. Other than segmentation, a parametric nature is useful to predict occluded objects, and also more convenient for shape analysis. Although Stardist worked well for cell nuclei segmentation (which are generally elliptical and star convex shapes), it performs badly in more complex data\par
One promising method is cellpose~\cite{Stringer2020} which aim to provide a general method to solve the segmentation problem (for different modalities, shapes, sizes using a single model), it outperformed Mask RCNN and even Stardist.\\

In this project we review recent works on the subject of cell segmentation, we then implement SplineDist~\cite{Mandal2020} an approach that uses parametric spline for segmenting cells without requiring further post-processing. We show that combining Multistar~\cite{Walter2020} approach for dealing with overlapping objects with the Spline curve generated by SplineDist offer more accurate predictions with minimal change to the Stardist Approach using less control points and without suffering from the star convexity constraint.\\

This report is organized as follows. We present the related works and the techniques used in the literature. We introduce the B-spline framework. We further explain in detail the model used in SplineDist (as well as the improvements we propose called MultiSpline). Next, we show the codebases used for the comparisons. We then proceed with explanation of the experimentation plan. Finally, we review the results obtained and assess the effectiveness of our approach. We conclude our project with further demonstrations on different image modalities and further improvements for 3D segmentation.\\

% \section{Digital pathology}

% \section{Cell morphology}
% \section{Cell biophysics}




\end{document}
